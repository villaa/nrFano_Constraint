\documentclass[12pt]{book}
\usepackage[utf8]{inputenc}
\usepackage{graphicx}
\graphicspath{ {images/} }
\usepackage[margin=1in]{geometry}
\usepackage{setspace}
\doublespacing
\usepackage{amsmath}
\usepackage{amssymb}
\usepackage{notoccite}
\usepackage{subfig}



\begin{document}
    

\begin{enumerate}
    \item Find the true recoil energy. 
    \begin{itemize}
        \item Create a uniform distribution of energy between 10 and 150 keV.
        \item The true recoil energy is then randomly drawn for this distribution. 
    \end{itemize} 
  \item Calculate the average number of electron hole pairs produced $\Bar{N}$ based on the yield $Y$.
   \begin{itemize}
        \item $Y$ is the average yield calculated from Lindhard for a given recoil energy.  
    \end{itemize} 
  
  \item Randomly draw the number of electron hole pairs produced $N$
   \begin{itemize}
        \item $N$ is randomly drawn from a normal distribution with a mean of $\Bar{N}$ and a standard deviation $\sqrt{\Bar{N}F}$, where F is the fano factor. 
    \end{itemize} 

    \item Calculate $E_P$ and $E_Q$ based on $N$.
       \begin{itemize}
        \item $E_P$ and $E_Q$ are considered the "true" values. 
    \end{itemize} 
    
    \item Calculate detector resolutions $\sigma_p$ and $\sigma_q$ 
        \begin{itemize}
        \item  $\sigma_p(E_P)$ and $\sigma_q(E_Q)$ are based on the quantities found in Dan Jardin's note [ref].
        \end{itemize} 
    
    \item Smear $E_P$ and $E_Q$ with $\sigma_p$ and $\sigma_q$ to find $\Tilde{E_P}$ and $\Tilde{E_Q}$
        \begin{itemize}
        \item Create two normal distributions with means $E_P$ and $E_Q$ and standard deviations $\sigma_p$ and $\sigma_q$.
        \item Randomly draw from these distributions to find $\Tilde{E_P}$ and $\Tilde{E_Q}$ respectivly. 
        \end{itemize} 
    \item Calculate the 'measured' recoil energy $\Tilde{E_r}$ using $\Tilde{E_P}$ and $\Tilde{E_Q}$. 
    \item Calculate the 'measured' yield $\Tilde{Y}$
    
    
\end{enumerate}

\section{Expected Yield Algorithm }
The following algorithm is to calculate the expected amount of energy given to the electronic system as a function of energy for electron and nuclear recoils. 

\begin{enumerate}
    \item Find the true recoil energy. 
    \begin{itemize}
        \item Create a uniform distribution of energy between 10 and 150 keV.
        \item The true recoil energy is then randomly drawn for this distribution. 
    \end{itemize} 
    
    \item Calculate the average number of electron hole pairs produced $\Bar{N}$ based on the yield $Y$.
        \begin{itemize}
            \item $Y$ is the average yield calculated from Lindhard for a given recoil energy ( for electron recoils $Y=1$.)
        \end{itemize}
    
    \item Calculate $\Bar{E_P}$ and $\Bar{E_Q}$ based on $\Bar{N}$
    
    \item Calculate $\sigma_P(\Bar{E_P})$ and $\sigma_Q(\Bar{E_Q})$ based on Dan Jardin's note [ref]
    \newpage
    \item Add $\epsilon^2 F\Bar{N}$ to $\sigma_Q^2$ to get $\Tilde{\sigma_Q^2} = \sqrt{\sigma_Q(\Bar{E_Q})^2 + \epsilon^2 F\Bar{N}}$
        \begin{itemize}
            \item The fano factor $F$ is added here, as we are not varying the number of electron hole pairs created.
        \end{itemize}
        
    \item Add $(eV)^2 F\Bar{N}$ to $\Tilde{\sigma_P^2}$
    
    \item 
    

\end{enumerate}

\end{document}